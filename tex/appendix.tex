
\chapter{\pulse: Supplementary Materials}

%\renewcommand\thesection{\Alph{section}.}
%\renewcommand\thesubsection{\Alph{section}.\arabic{subsection}}
%\setcounter {subfigure} {0}
%\setcounter {figure} {0}
%\setcounter {section} {0}
%\setcounter {page} {1}

\section{Multiplexing $M+N$ Iterator Executions for Maximizing Pipeline Utilization}

We claimed in \S\ref{ssec:pulsearchitecture} that if $t_c = \eta \cdot t_d$ for all offloaded iterator executions, it is always possible to multiplex $m + n$ concurrent iterator executions and fully utilize all memory and logic pipelines. We prove our claim by providing a staggered scheduling algorithm (Algorithm~\ref{alg:scheduling}) that ensures such multiplexing across $m+n$ iterator executions. The scheduler processes $m+n$ iterator execution requests, assigning each a memory pipeline, a logic pipeline, and staggered start times. These requests are then executed in the respective memory pipelines. Through this staggered scheduling approach, \jiffy fully utilizes the $n$ memory pipelines and $m$ logic pipelines, ensuring no resources are wasted. Note that this algorithm is a simplified version to illustrate the potential for full pipeline saturation under the given condition. \jiffy's scheduler implements a real-time algorithm to multiplex incoming requests on the fly.


\begin{algorithm}
\caption{Staggered-Scheduling}
\label{alg:scheduling}
\begin{algorithmic}[1]
\State $m, n \gets$ number of logic, memory pipelines
\State $L_i, M_j \gets$ $i^{th}$ logic pipeline, $j^{th}$ memory pipeline
\State $t_d \gets$ data fetch time per pointer traversal iteration

\While{true}
    %\State Increment $clock$
    \State Dequeue $n + m$ requests from network stack
    \For{$i \gets 1$ \textbf{to} $m + n$}
        \State Assign request $R_i$ to ($M_{i \bmod n}$,  $L_{i \bmod m}$)
        \State Schedule $R_i$ to start at time $(i-1) \cdot \frac{t_d}{n}$
        
        
    \EndFor
    
    \State Start requests as scheduled at memory pipelines
    
\EndWhile
\end{algorithmic}
\end{algorithm}


\section{\pulse Empirical Analysis}
Prior studies have shown that real-world data-centric cloud applications spend a significant fraction of time traversing pointers, as summarized in Fig.~\ref{fig:sup_motivation}.

\begin{figure}[b]
    \centering
    \footnotesize
     \subfigure[Survey from prior studies]{
                \begin{tabular}[b]{c|c} 
                    \textbf{Application} & \specialcell{\textbf{\% of time spent}\\\textbf{in pointer traversal}} \\ \hline
                    GraphChi~\cite{graphchi} & $\sim 93\%$ \\ \hline
                    MonetDB\cite{monetdb} & $70\%-97\%$ \\ \hline
                    GC in Spark~\cite{spark} & $\sim 72$\% \\ \hline
                    VoltDB~\cite{voltdb} & Up to $49.55$\% \\ \hline
                    MemC3~\cite{memc3} & Up to $21.15$\% \\ \hline
                    DBx1000 \cite{db1000} & $\sim 9$\% \\\hline
                    Memcached\cite{memcached} & $\sim 7$\% \\ \hline\hline
                \end{tabular} 
    }
            \label{tab:sup_motivation}
    \caption[Time cloud applications spend in pointer traversals]{\textbf{Time cloud applications spend in pointer traversals} based on prior studies}
    \label{fig:sup_motivation}
\end{figure}

\section{\pulse Supported Data Structures}

We adapt $13$ data structures across $4$ popular open-sourced libraries to \pulse's iterator abstraction (\S\ref{ssec:pulseinterface}). In particular, we outline how the data structure implementations for certain operations can be expressed using \code{init()}, \code{next()}, and \code{end()}. For simplicity and readability, (i) we assume that the data structure developer defines a macro, \code{SP\_PTR\_(variable\_name)}, as the address of the variable resides on the \code{scratch\_pad}, and (ii) we omit obvious type conversions for de-referenced pointers. 

We analyze two widely used categories of data structures: lists and trees. In our analysis, we find that the top-level data structure APIs (\ie, the APIs used by applications) use the same base function under the hood. For instance, list and forward list in the STL library share the same internal function, \code{std::find()}. We summarize our findings in Table \ref{table:extra-apps-2}, including the data structure libraries, their category, the top-level data structure APIs, and the internal base function.

\paragraphb{List structures} Our surveyed list structures already follow the execution flow of \pulse iterator: \code{init()}, \code{next()}, and \code{end()}.

These data structures generally have compute-intensive \code{end()} functions to check multiple termination conditions, while their \code{next()} function simply dereferences a single pointer to the next node. Listing \ref{lst:list} and Listing \ref{lst:list_mod} demonstrate a linked list with two termination conditions: (i) value is found or (ii) search reaches the end. To indicate which condition is met, a special flag (\eg, \code{KEY\_NOT\_FOUND}) is written on the \code{scratch\_pad}. Listing~\ref{lst:bimap} and Listing~\ref{lst:bimap_mod} describe a bitmap that uses a hashtable internally, where colliding entries are stored in linked lists within the same bucket. As such, the \pulse iterator interface resembles that of \code{std::list} quite closely.

\paragraphb{Tree-like data structures}
Compared to list structures, tree data structures require more computation in the \code{next()} function, as the next pointer is determined based on the value in the child node. For instance, in \code{Btree} (Listing \ref{lst:btree}, \ref{lst:btree_mod}), the next function iterates through internal node keys, comparing them to the search key. Interestingly, \code{std::map} (Listing \ref{lst:map}, \ref{lst:map_mod}) and Boost AVL trees (Listing \ref{lst:avltree}, \ref{lst:avltree_mod}) share the same offload function structure, with only minor implementation and naming differences.

\begin{table*}[!h]
  \centering
  \captionof{table}[Additional data structure supported by \pulse]{\small Additional data structure supported by \pulse.}
  \scriptsize
  \begin{tabular}{L{.12\textwidth}|L{.06\textwidth}|L{.07\textwidth}|L{.1\textwidth}|L{.29\textwidth}|L{.09\textwidth}|L{.09\textwidth}}
    \hline
    {\bf Data Structure} & {\bf Category} & {\bf Library} & {\bf Data structure API} & {\bf Internal function} & {\bf Original code} & {\bf \pulse code}\\\hline
    \hline
    List & \multirow{5}{*}{List} & \multirow{2}{*}{STL} & \multirow{2}{*}{\texttt{std::find()}} & \multirow{2}{*}{\texttt{std::find()}} &\multirow{2}{*}{Listing~\ref{lst:list}} & \multirow{2}{*}{Listing~\ref{lst:list_mod}}  \\\cline{1-1}
    Forward list & & &  & & & \\\cline{1-1}\cline{3-7}
    Bimap &  & \multirow{3}{*}{Boost} &  \multirow{3}{*}{\texttt{find()}} & \multirow{3}{*}{\texttt{find()}} &\multirow{3}{*}{Listing \ref{lst:bimap}} & \multirow{3}{*}{Listing \ref{lst:bimap_mod}}\\\cline{1-1}
    Unordered map  &  &  &  &  & & \\\cline{1-1}
    Unordered set  &  &  &  &  & & \\\hline
    Btree & \multirow{8}{*}{Tree} & Google & \multirow{8}{*}{\texttt{find()}} & \makecell[l]{\texttt{internal\_locate\_plain}\\\texttt{\_compare(key, iter)}} & Listing \ref{lst:btree} & Listing \ref{lst:btree_mod}\\\cline{1-1}\cline{3-3}\cline{5-7}
    Map & & \multirow{4}{*}{STL} & & \multirow{4}{*}{\texttt{\_M\_lower\_bound(x, y, key)}} &\multirow{4}{*}{Listing \ref{lst:map}} & \multirow{4}{*}{Listing \ref{lst:map_mod}}\\\cline{1-1}
    Set & & & & &  & \\\cline{1-1}
    Multimap & & & & & & \\\cline{1-1}
    Multiset & & & & & & \\\cline{1-1}\cline{3-3}\cline{5-7}
    AVL tree & & \multirow{3}{*}{Boost} & & \multirow{3}{*}{\texttt{lower\_bound\_loop(x, y, key)}} &\multirow{3}{*}{Listing \ref{lst:avltree}} & \multirow{3}{*}{Listing~\ref{lst:avltree_mod}}\\\cline{1-1}
    Splay tree & & & & & & \\\cline{1-1}
    Scapegoat tree & & & & & & \\
    \hline
  \end{tabular}
  \label{table:extra-apps-2}
\end{table*}
\setcounter {lstlisting} {0}
\setcounter {table} {1}



% STL
\lstset{frame=tb,
  xleftmargin=0cm,
  linewidth=0.95\textwidth
}
\captionsetup[lstlisting]{style=centered_lstlisting}


\newpage
\subsection{List data structure in STL library}
\centering
\begin{lstlisting}[caption={C++ STL realization for \code{std::find()}},label={lst:list}]
struct node {
    value_type value;
    struct node* next;
};
node* find(node* first, node* last, const value_type& value) {
    for (; first != last; first=first->next)
        if (first->value == value)
            return first;
    return last;
}
\end{lstlisting}
\newpage

\begin{lstlisting}[caption={\pulse realization for \code{std::find()}},label={lst:list_mod}]
class list_find : chase_iterator {
    init(void *value, void* first) {sssec:pulsebreakdown
    }
    void* next() {
        return cur_ptr->next;
    }
    bool end() {
        if (*SP_PTR_VALUE == cur_ptr->value) {
            *SP_PTR_RETURN = cur_ptr;  
            return true;
        }
        if (cur_ptr->next == NULL) {
            *SP_PTR_RETURN = KEY_NOT_FOUND;  
            return true;
        }
        return false;
    }
}
\end{lstlisting}
\newpage


\subsection{List data structure in Boost library}
\centering
\begin{lstlisting}[caption={Boost realization for \code{bimap::find()}},label={lst:bimap}]
struct node {
    key_type key;
    struct node* next;
    value_type value;
};
void* find(const key_type& key, const hash_type& hash) const {
    std::size_t buc = buckets.position(hash(key));
    node_ptr start = buckets.at(buc)
    for(node_ptr x = start; x != NULL; x = x->next) {
        if(key == x->key) {
            return x;
        }
    }
    return NULL;
}
\end{lstlisting}

\newpage

\begin{lstlisting}[caption={\pulse realization for \code{bimap::find()}},label={lst:bimap_mod}]
class bimap_find : chase_iterator {
public:
    key_type key;
    init(void *key, void* start) {
        *SP_PTR_KEY = key;
        cur_ptr = start;
    }
    void* next() {
        return cur_ptr->next;
    }
    bool end() {
        if (*SP_PTR_KEY == cur_ptr->key) {
            *SP_PTR_RETURN = cur_ptr;
            return true;
        }
        if (cur_ptr->next == NULL) {
            *SP_PTR_RETURN = NULL;  
            return true;
        }
        return false;
    }
}
\end{lstlisting}
\newpage

\subsection{Tree data structure in Google library}
\centering
\begin{lstlisting}[caption={Google realization for\\\code{btree::internal\_locate\_plain\_compare()}},label={lst:btree}]
#define kNodeValues 8
struct btree_node {
    bool is_leaf;    
    int num_keys;
    key_type keys[kNodeValues];
    btree_node* child[kNodeValues + 1];
};
IterType btree::internal_locate_plain_compare(const key_type &key, IterType iter) const {
    for (;;) { 
        int i;
        for(int i = 0; i < iter->num_keys; i++) {
            if(key <= iter->keys[i]) {
                break;
            }
        }
        if (iter.node->is_leaf) {
            break;
        }
        iter.node = iter.node->child(i);
    }
    return iter;
}
\end{lstlisting}
\newpage

\begin{lstlisting}[caption={\pulse realization for\\\code{btree::internal\_locate\_plain\_compare()}},label={lst:btree_mod}]
class btree_find_unique : chase_iterator {
    init(void *key, void* iter) {
        *SP_PTR_KEY = key;
        cur_ptr = iter;
    }
  
    void* next() {
        *SP_PTR_I = 0;
        for(; *SP_PTR_I < cur_ptr->num_keys; *SP_PTR_I++) {
            if(*SP_PTR_KEY <= cur_ptr->keys[*SP_PTR_I]) 
            {
                break;
            }
        }
        cur_ptr = cur_ptr->child[*SP_PTR_I];
    }
  
    bool end() {
        if(cur_ptr->is_leaf) {
            *SP_PTR_RETURN = cur_ptr;
            return true;
        } else {
            return false;
        }
    }
}
\end{lstlisting}

\newpage
\subsection{Tree data structure in STL library}
\centering
\begin{lstlisting}[caption={C++ STL realization for \code{map::find()}},label={lst:map}]
struct node {
    key_type key;
    node* left;
    node* right;
};

_M_lower_bound(node* x, node* y, const key_type& key)
{
    while (x != 0) {
        if (x->key <= key) {
            y = x; 
            x = x->left;
        } else {
            x = x->right;
        }
    }
    return y;
}
\end{lstlisting}

\newpage
\begin{lstlisting}[caption={\pulse realization for \code{map::find()}},label={lst:map_mod}]
class map_find : chase_iterator {
    init(void *key, void* x, void* y) {
        *SP_PTR_KEY = key;
        *SP_PTR_Y = y;
        cur_ptr = x;
    }
  
    void* next() {
        if (cur_ptr->key <= *SP_PTR_KEY) {
            *SP_PTR_Y = cur_ptr;  
            cur_ptr = cur_ptr->left;
        } else {
            cur_ptr= cur_ptr->right;
        }
        return cur_ptr->left;
    }
  
    bool end() {
        if (cur_ptr == NULL) {
            *SP_PTR_RETURN = *SP_PTR_Y;  
            return true;
        } else {
            return false;
        }
    }
}
\end{lstlisting}
\newpage

% Boost
\subsection{Tree data structure in Boost library}
\centering
\begin{lstlisting}[caption={Boost realization for \code{avltree::find()}},label={lst:avltree}]
static node_ptr lower_bound_loop
(node_ptr x, node_ptr y, const KeyType &key)
{
    while(x){
        if(x->key >= key)) {
            x = x->right;
        }
        else{
            y = x;
            x = x->left;
        }
    }
    return y;
}
\end{lstlisting}
\newpage

\begin{lstlisting}[caption={\pulse realization for \code{avltree::find()}},label={lst:avltree_mod}]
class avltree_find : chase_iterator {
public:
    key_type key;
    void* y;
  
    init(void *key, void* x, void* y) {
        *SP_PTR_KEY = key;
        *SP_PTR_Y = y;
        cur_ptr = x;
    }
  
    void* next() {
        if(cur_ptr->key >= *SP_PTR_KEY) {
            cur_ptr = cur_ptr->right;
        }
        else{
            *SP_PTR_Y = cur_ptr;
            cur_ptr = cur_ptr->left;
        }
    }
  
    bool end() {
        // The result is already stored at SP_PTR_Y
        if(cur_ptr == NULL) {
            return true;
        } else {
            return false;
        }
    }
}
\end{lstlisting}
