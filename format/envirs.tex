% BEGIN Theorem environment settings
\theoremstyle{plain}
\newtheorem{theorem}{Theorem}
\newtheorem{myTheorem}{Theorem}[section]
\newtheorem{myLemma}[myTheorem]{Lemma}
\newtheorem{myCorollary}[myTheorem]{Corollary}
\newtheorem{myClaim}{Claim}[section]
\newtheorem{myRemark}{Remark}
\newtheorem{myExample}{Example}[section]
\newtheorem{myConjecture}{Conjecture}[section]
\newtheorem{myDef}{Definition}[section]

\newtheorem*{myNote}{Note}

\definecolor{boxclr}{gray}{0.9}
\newenvironment{colframe}{%
  \begin{Sbox}
    \begin{minipage}
      {0.96\columnwidth}
    }{%
    \end{minipage}
  \end{Sbox}
  \begin{center}
    \colorbox{boxclr}{\TheSbox}
  \end{center}
}

\renewcommand{\figurename}{Fig.}

\pgfdeclarelayer{background}    % declare background layer
\pgfsetlayers{background,main}  % set the order of the layers (main is the standard layer)

\makeatletter
\newcommand{\thickhline}{%
    \noalign {\ifnum 0=`}\fi \hrule height 0.8pt
    \futurelet \reserved@a \@xhline
}
\newcolumntype{"}{@{\vrule width 0.8pt}}
\newcolumntype{[}{@{\vrule width 0.8pt\hskip\tabcolsep}}
\newcolumntype{]}{@{\hskip\tabcolsep\vrule width 0.8pt}}
\newcolumntype{!}{@{\hskip\tabcolsep\vrule width 0.8pt\hskip\tabcolsep}}
\makeatother

\newcommand{\cppsnippet}[1]{%
  \begin{lstlisting}[gobble=4]
    #1
  \end{lstlisting}
}

\tikzset{
    position/.style args={#1:#2 from #3}{
        at=(#3.#1), anchor=#1+180, shift=(#1:#2)
    }
}

\tikzset{
  half fill/.style 2 args={fill=#2, path picture={
    \fill[#1, sharp corners] (path picture bounding box.west) --
                         (path picture bounding box.east) --
                         (path picture bounding box.south east) --
                         (path picture bounding box.south west) -- cycle;}},
}

\tikzset{
  nil fill/.style 2 args={fill=#2, path picture={
    \fill[#1, sharp corners] (path picture bounding box.155) --
                         (path picture bounding box.25) --
                         (path picture bounding box.south east) --
                         (path picture bounding box.south west) -- cycle;}},
}

\tikzset{
  almost fill/.style 2 args={fill=#2, path picture={
    \fill[#1, sharp corners] (path picture bounding box.205) --
                         (path picture bounding box.335) --
                         (path picture bounding box.south east) --
                         (path picture bounding box.south west) -- cycle;}},
}

\definecolor{oldcolor}{HTML}{c66541}

\newcommand*\circled[1]{\tikz[baseline=(char.base)]{
            \node[shape=circle,draw,inner sep=0.5pt] (char) {\small #1};}}

\newcommand{\specialcellc}[2][c]{%
  \begin{tabular}[#1]{@{}c@{}}#2\end{tabular}}
  
\newcommand*\numcircledtikz[1]{\tikz[baseline=(char.base)]{\node[shape=circle,draw,inner sep=0.75pt,fill=white] (char) {#1};}}
% Proofs get merged into text, with a "Proof" flag and box at the end
\theoremstyle{nonumberplain}
\newcommand{\proofsubhead}[1]{\noindent{\normalfont\bfseries\boldmath#1}}

\newtheorem{myProof}{Proof}
\newtheorem{myProofSketch}{Proof Sketch}
% END Theorem environment settings

\algdef{SE}[DOWHILE]{Do}{doWhile}{\algorithmicdo}[1]{\algorithmicwhile\ #1}%

\newenvironment{denseitemize}{
\begin{itemize}[topsep=2pt, partopsep=0pt, leftmargin=1.5em]
  \setlength{\itemsep}{4pt}
  \setlength{\parskip}{0pt}
  \setlength{\parsep}{0pt}
}{\end{itemize}}

\newenvironment{denseenum}{
\begin{enumerate}[topsep=2pt, partopsep=0pt, leftmargin=1.5em]
  \setlength{\itemsep}{4pt}
  \setlength{\parskip}{0pt}
  \setlength{\parsep}{0pt}
}{\end{enumerate}}

\newcommand{\rcl}[1]{{\color{red}{$\gets$ #1}}}
\newcommand{\rcr}[1]{{\color{red}{$\to$ #1}}}
\newcommand{\rqc}[1]{{\color{blue}{#1}}}
\newcommand{\gc}[1]{{\color{ForestGreen}{#1}}}
\newcommand{\new}[1]{{\color{blue}{#1}}}

\newcommand{\cut}[1]{}
\renewcommand{\ttdefault}{txtt}
\newcommand{\paragraphb}[1]{\vspace{0.075in}\noindent{\bf #1.}}
\newcommand{\paragrapha}[1]{\vspace{0.075in}\noindent{\bf #1}}
\newcommand{\paragraphc}[1]{\vspace{0.075in}\noindent{\em #1}}
\newcommand{\todo}[1]{\textcolor{Red}{\{#1\}}}
\newcommand{\crdy}[1]{\textcolor{burntorange}{\textit{<#1>}}}

\newcommand{\hlc}[2][yellow]{{\sethlcolor{#1} \hl{#2}} }
\newcommand\yupeng[1]{\hlc[yellow]{YT: -- #1 --}}
\newcommand\yupengpost[1]{\textcolor{red}{#1}}
\newcommand\reviewer[1]{\textcolor{blue}{#1}}
\colorlet{soulgreen}{green!30}
\newcommand\slee[1]{\hlc[soulgreen]{SL: -- #1 --}}
\colorlet{soulred}{blue!20}
\newcommand\anurag[1]{\hlc[soulred]{AK: -- #1 --}}
\newcommand\yupenga[1]{\hlc[lightgray]{YT: -- #1 --}}
\newcommand\sleea[1]{\hlc[lightgray]{SL: -- #1 --}}
\newcommand\anuraga[1]{\hlc[lightgray]{AK: -- #1 --}}

\newcommand\readinggroup[1]{\hlc[soulpink]{RG: -- #1 --}}
\colorlet{soulpink}{red!20}

\tikzset{ 
table/.style={
  matrix of nodes,
  row sep=-\pgflinewidth,
  column sep=-\pgflinewidth,
  nodes={rectangle,thick,draw=black,text width={},align=center,font=\small},
  text depth=0.25ex,
  text height=1.25ex,
  nodes in empty cells
},
map/.style={
  matrix of nodes,
  row sep=-\pgflinewidth,
  column sep=-\pgflinewidth,
  nodes={rectangle,draw=black,text width=5em,align=center,font=\small},
  text depth=0.25ex,
  text height=1.25ex,
  nodes in empty cells
},
bigmap/.style={
  matrix of nodes,
  row sep=-\pgflinewidth,
  column sep=-\pgflinewidth,
  nodes={rectangle,draw=black,text width=26em,align=center,font=\small},
  text depth=0.25ex,
  text height=1.25ex,
  nodes in empty cells
},
memcell/.style={
  draw, 
  very thick, 
  text width=0.25em, 
  text height=0.25em
},
}

\tikzstyle{startstop} = [rectangle, rounded corners, minimum width=3em, minimum height=1em,text centered, draw=black, fill=red!30]
\tikzstyle{io} = [trapezium, trapezium left angle=70, trapezium right angle=120, minimum width=2.5em, minimum height=1em, text centered, draw=black, fill=blue!30]
\tikzstyle{process} = [rectangle, minimum width=1.5em, minimum height=1em, align=center, draw=black, fill=gray!30]
\tikzstyle{decision} = [diamond, minimum width=3em, minimum height=1em, align=center, draw=black, fill=SkyBlue!30]
\tikzstyle{arrow} = [thick,->,>=stealth]
\tikzstyle{monolog} = [fill=SkyBlue!30]

\tikzset{
  % style to apply some styles to each segment of a path
  on each segment/.style={
    decorate,
    decoration={
      show path construction,
      moveto code={},
      lineto code={
        \path [#1]
        (\tikzinputsegmentfirst) -- (\tikzinputsegmentlast);
      },
      curveto code={
        \path [#1] (\tikzinputsegmentfirst)
        .. controls
        (\tikzinputsegmentsupporta) and (\tikzinputsegmentsupportb)
        ..
        (\tikzinputsegmentlast);
      },
      closepath code={
        \path [#1]
        (\tikzinputsegmentfirst) -- (\tikzinputsegmentlast);
      },
    },
  },
  % style to add an arrow in the middle of a path
  mid arrow/.style={postaction={decorate,decoration={
        markings,
        mark=at position .5 with {\arrow[#1]{stealth}}
      }}},
}

\newcommand{\specialcell}[2][l]{%
  \begin{tabular}[#1]{@{}l@{}}#2\end{tabular}}
    
\renewcommand{\UrlFont}{\small\tt}

\newcommand{\cmark}{\color{green}\ding{51}}%
\newcommand{\xmark}{\color{red}\ding{55}}%

\newcolumntype{L}[1]{>{\RaggedRight\hspace{0pt}}p{#1}}
\newcolumntype{R}[1]{>{\RaggedLeft\hspace{0pt}}p{#1}}

\algnewcommand{\IIf}[1]{\State\algorithmicif\ #1\ \algorithmicthen}%
\algnewcommand{\EndIIf}{\unskip\ }%
\algdef{SE}[DOWHILE]{Do}{doWhile}{\algorithmicdo}[1]{\algorithmicwhile\ #1}%
\algnewcommand\algorithmicforeach{\textbf{for each}}%
\algdef{S}[FOR]{ForEach}[1]{\algorithmicforeach\ #1\ \algorithmicdo}%

\newcommand{\smalltitle}[1]{\vspace{4pt} \noindent \textbf{#1}\hspace{1.4pt}}

\newcommand{\code}[1]{{\fontsize{9}{11}\selectfont\texttt{#1}}}
\newcommand{\smallcode}[1]{{\fontsize{6.5}{11}\selectfont\texttt{#1}}}

\newcommand{\draft}[1]{{\textcolor{red}{#1}}}
\newenvironment{icompact}{
\begin{itemize}[topsep=1.5pt, partopsep=0pt, leftmargin=1em]
  \setlength{\itemsep}{1pt}
  \setlength{\parskip}{0pt}
 \setlength{\parsep}{0pt}
}{\end{itemize}}

